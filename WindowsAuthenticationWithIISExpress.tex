option-1:

edit \My Documents\IISExpress\config\applicationhost.config file and enable windowsAuthentication, i.e:

<system.webServer>
...
  <security>
...
    <authentication>
      <windowsAuthentication enabled="true" />
    </authentication>
...
  </security>
...
</system.webServer>

Unlock windowsAuthentication section in \My Documents\IISExpress\config\applicationhost.config as follows

<add name="WindowsAuthenticationModule" lockItem="false" />
Alter override settings for the required authentication types to 'Allow'

<sectionGroup name="security">
    ...
    <sectionGroup name="system.webServer">
        ...
        <sectionGroup name="authentication">
            <section name="anonymousAuthentication" overrideModeDefault="Allow" />
            ...
            <section name="windowsAuthentication" overrideModeDefault="Allow" />
    </sectionGroup>
</sectionGroup>

	
It looks like you solved your own question! Good on you. In addition to this post helping me I found the following to be SUPER helpful in configuring my IIS Express.

IIS Express Windows Authentication

Edit: I've copied the important information from the associated link in case it dies. This is completely from user vikomall

option-1:

edit \My Documents\IISExpress\config\applicationhost.config file and enable windowsAuthentication, i.e:

<system.webServer>
...
  <security>
...
    <authentication>
      <windowsAuthentication enabled="true" />
    </authentication>
...
  </security>
...
</system.webServer>
option-2:

Unlock windowsAuthentication section in \My Documents\IISExpress\config\applicationhost.config as follows

<add name="WindowsAuthenticationModule" lockItem="false" />
Alter override settings for the required authentication types to 'Allow'

<sectionGroup name="security">
    ...
    <sectionGroup name="system.webServer">
        ...
        <sectionGroup name="authentication">
            <section name="anonymousAuthentication" overrideModeDefault="Allow" />
            ...
            <section name="windowsAuthentication" overrideModeDefault="Allow" />
    </sectionGroup>
</sectionGroup>
Add following in the application's web.config

<?xml version="1.0" encoding="UTF-8"?>
<configuration>
    <system.webServer>
      <security>
        <authentication>
          <windowsAuthentication enabled="true" />v
        </authentication>
      </security>
    </system.webServer>
</configuration>

<authentication>

            <anonymousAuthentication enabled="false" userName="" />
			
Be careful with the applicationhost.config modifications - in Visual Studio 2015 I've found that it
resides in the local project directory.

For example:

DRIVE:\MYPROJECT\.vs\config\applicationhost.config
